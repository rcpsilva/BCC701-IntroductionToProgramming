\documentclass{article}
\usepackage[utf8]{inputenc}
\usepackage{amsmath}

\title{Exercícios Práticos de Programação}
\author{BCC701}
\date{2024}

\begin{document}

\maketitle

\section*{Capítulo 1: Introdução}
\begin{itemize}
    \item \textbf{Exercício 1: Algoritmo de Conversão de Temperatura}
    \begin{itemize}
        \item Escreva um programa que leia uma temperatura em graus Celsius e a converta para Fahrenheit utilizando a fórmula \( F = \frac{C \times 9}{5} + 32 \).
    \end{itemize}
\end{itemize}

\section*{Capítulo 2: Variáveis e Expressões}
\begin{itemize}
    \item \textbf{Exercício 2: Soma de Dois Números}
    \begin{itemize}
        \item Escreva um programa que declare duas variáveis inteiras, atribua valores a elas, e imprima a soma desses valores.
    \end{itemize}
    \item \textbf{Exercício 3: Área de um Retângulo}
    \begin{itemize}
        \item Escreva um programa que leia a largura e a altura de um retângulo e calcule sua área.
    \end{itemize}
    \item \textbf{Exercício 4: Área da circunferência}
    \begin{itemize}
        \item Escreva um programa que leia a largura e a altura de um retângulo e calcule sua área.
    \end{itemize}
    \item \textbf{Exercício 5: Concatenar Strings}
    \begin{itemize}
        \item Escreva um programa que declare duas variáveis do tipo string e concatene seus valores. Imprima o resultado.
    \end{itemize}
\end{itemize}

\section*{Capítulo 3: Entrada e Saída}
\begin{itemize}
    \item \textbf{Exercício 6: Entrada de Dados}
    \begin{itemize}
        \item Escreva um programa que leia o nome e a idade do usuário e imprima uma mensagem dizendo "Olá, [nome]! Você tem [idade] anos."
    \end{itemize}
    \item \textbf{Exercício 7: Cálculo do IMC}
    \begin{itemize}
        \item Escreva um programa que leia o peso e a altura do usuário e calcule o Índice de Massa Corporal (IMC) usando a fórmula \( \text{IMC} = \frac{\text{peso}}{\text{altura}^2} \). Imprima o resultado.
    \end{itemize}
\end{itemize}

\section*{Capítulo 4: Estrutura de Decisão}
\begin{itemize}
    \item \textbf{Exercício 8: Número Par ou Ímpar}
    \begin{itemize}
        \item Escreva um programa que leia um número inteiro e imprima se ele é par ou ímpar.
    \end{itemize}
    \item \textbf{Exercício 9: Verificação de Maioridade}
    \begin{itemize}
        \item Escreva um programa que leia a idade de uma pessoa e imprima se ela é maior de idade (18 anos ou mais) ou menor de idade.
    \end{itemize}
\end{itemize}
    
\section*{Capítulo 4.1: Estrutura de Decisão Aninhadas}
    \begin{itemize}
        \item \textbf{Exercício 10: Maior de Três Números}
        \begin{itemize}
            \item Escreva um programa que leia três números inteiros e imprima o maior deles.
        \end{itemize}
        \item \textbf{Exercício 11: Classificação de Notas}
        \begin{itemize}
            \item Escreva um programa que leia a nota de um aluno e imprima a classificação de acordo com a seguinte tabela:
            \begin{itemize}
                \item Nota \(\geq\) 90: A
                \item Nota \(\geq\) 80 e $<$ 90: B
                \item Nota \(\geq\) 70 e $<$ 80: C
                \item Nota \(\geq\) 60 e $<$ 70: D
                \item Nota $<$ 60: F
            \end{itemize}
        \end{itemize}
        \item \textbf{Exercício 12: Pedra, Papel e Tesoura}
        \begin{itemize}
            \item Escreva um programa para o jogo pedra, papel e tesoura.
        \end{itemize}
    \end{itemize}

\end{document}
