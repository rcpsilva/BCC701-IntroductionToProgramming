\documentclass{article}
\usepackage[utf8]{inputenc}
\usepackage{amsmath}
\usepackage{geometry}

\geometry{
  a4paper,
  left=2cm,
  right=2cm,
  top=2cm,
  bottom=2cm
}

\title{Exercícios Práticos de Programação}
\author{BCC701}
\date{2024}

\begin{document}

\maketitle

\subsection*{Estruturas de Decisão Simples}

\begin{enumerate}
    \item Escreva um programa que leia um número inteiro e verifique se ele é positivo. Imprima uma mensagem indicando se o número é positivo ou não.
    
    \item Crie um programa que leia a idade de uma pessoa e verifique se ela é maior de idade (18 anos ou mais). Imprima uma mensagem indicando se a pessoa é maior de idade ou não.
    
    \item Desenvolva um programa que leia um número inteiro e verifique se ele é par ou ímpar. Imprima uma mensagem indicando o resultado.
    
    \item Faça um programa que leia a temperatura em graus Celsius e verifique se está acima de 30 graus. Imprima uma mensagem indicando se está quente ou não.
    
    \item Escreva um programa que leia um valor e verifique se ele é maior que 100. Imprima uma mensagem indicando se o valor é maior que 100 ou não.
    
    \item Crie um programa que leia um número inteiro e verifique se ele é múltiplo de 5. Imprima uma mensagem indicando se o número é múltiplo de 5 ou não.
    
    \item Desenvolva um programa que leia o saldo de uma conta bancária e verifique se ele está negativo. Imprima uma mensagem indicando se o saldo está negativo ou não.
    
    \item Faça um programa que leia dois números inteiros e verifique se o primeiro é maior que o segundo. Imprima uma mensagem indicando o resultado.
    
    \item Escreva um programa que leia uma nota de 0 a 10 e verifique se ela é maior ou igual a 7. Imprima uma mensagem indicando se o aluno foi aprovado ou não.
    
    \item Crie um programa que leia a altura de uma pessoa e verifique se ela é maior ou igual a 1,70 metros. Imprima uma mensagem indicando se a pessoa é alta ou não.
\end{enumerate}

\subsection*{Estruturas de Decisão Aninhadas}

\begin{enumerate}
    \item Escreva um programa que leia três números inteiros e determine o maior deles. Imprima o resultado.
    
    \item Crie um programa que leia a idade de uma pessoa e verifique se ela é maior de idade. Se for maior de idade, verifique se ela tem mais de 65 anos. Imprima uma mensagem indicando se a pessoa é maior de idade e se é idosa.
    
    \item Desenvolva um programa que leia um número inteiro e verifique se ele é positivo. Se for positivo, verifique se é par ou ímpar. Imprima uma mensagem indicando o resultado.
    
    \item Faça um programa que leia a temperatura em graus Celsius e verifique se está acima de 30 graus. Se estiver, verifique se está acima de 40 graus. Imprima uma mensagem indicando se está quente ou muito quente.
    
    \item Escreva um programa que leia um valor e verifique se ele é maior que 100. Se for, verifique se é maior que 200. Imprima uma mensagem indicando se o valor é maior que 100 ou 200.
    
    \item Crie um programa que leia um número inteiro e verifique se ele é múltiplo de 3. Se for, verifique se é múltiplo de 5. Imprima uma mensagem indicando se o número é múltiplo de 3 e 5.
    
    \item Desenvolva um programa que leia o saldo de uma conta bancária e verifique se ele está negativo. Se estiver, verifique se está abaixo de -500. Imprima uma mensagem indicando se o saldo está negativo e se está muito negativo.
    
    \item Faça um programa que leia dois números inteiros e verifique se o primeiro é maior que o segundo. Se for, verifique se a diferença entre eles é maior que 10. Imprima uma mensagem indicando o resultado.
    
    \item Escreva um programa que leia uma nota de 0 a 10 e verifique se ela é maior ou igual a 7. Se for, verifique se é maior ou igual a 9. Imprima uma mensagem indicando se o aluno foi aprovado ou aprovado com distinção.
    
    \item Crie um programa que leia a altura de uma pessoa e verifique se ela é maior ou igual a 1,70 metros. Se for, verifique se é maior ou igual a 1,90 metros. Imprima uma mensagem indicando se a pessoa é alta ou muito alta.
\end{enumerate}

\end{document}
